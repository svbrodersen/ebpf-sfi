\section{Implementation}
\subsection{Prerequisite}
\begin{lstlisting}[language={haskell}, caption={Cheking prerequisites}, label={lst:pre}, escapechar=@]
checkPrerequisite :: ((Int, Instruction) -> Bool) -> LabeledProgram -> Maybe ()
checkPrerequisite p lprog =
  let res = any p lprog
   in (if res then Nothing else Just ())

-- Example use case for checking jump instructions.
checkJumpInstructions :: LabeledProgram -> Maybe ()
checkJumpInstructions lprog = checkPrerequisite checkInst lprog
 where
  checkInst (l, instr) = case instr of
    JCond _ _ _ off -> checkJump off l
    Jmp off -> checkJump off l
    -- Do not allow any calls to extern
    (Call _) -> True
    _ -> False
  checkJump off l =
    let
      res = all (\(i', _) -> i' == getLabel off l) lprog @\label{line:res}@
     in
      not res

checkPrerequisites :: LabeledProgram -> Maybe ()
checkPrerequisites lprog = do
  -- Registers 1,2,10 are restricted by definition of the assignment. register 9, 11 is used for guards
  _ <- checkRegisterUse lprog [1, 2, 9, 10, 11]
  _ <- checkJumpInstructions lprog
  -- Now that we have a program that is nice, we can just return
  return ()
\end{lstlisting}

The first item of the implementation is checking the prerequisites mentioned in
Section~\ref{sec:introduction}. For this I defined the helper function
\texttt{checkPrerequisite}, which can be seen in Listing~\ref{lst:pre}. The
function simply checks if the predicate is true for any of the statements in
the program. If it is, then we return \texttt{Nothing}, otherwise we return \texttt{Just ()}.

An example of the use case is the \texttt{checkJumpInstructions}. Within this
function, we define the predicate \texttt{checkInst}. If we have a jump
condition, then we have to check if there exists some other label in the
labeled program, which starts with the location we are attempting to jump to
(the \texttt{getLabel} takes the offset and the current label, and calculates
the label we would jump to). If the jump is illegal, then res would be False on
line~\ref{line:res}, so we negate res and return.

With this, we can use the functions with the do notation of the Maybe monad as
seen in \texttt{checkPrerequisites} in Listing~\ref{lst:pre}. If any of the
prerequisites return \texttt{Nothing}, then the entire function simply returns
\texttt{Nothing}.

\subsection{SFI Policy Algorithm}
\begin{lstlisting}[language={haskell}, caption={SFI algorithm}, label={lst:sfi}]
sfiAlgorithm' :: LabeledProgram -> Int -> Maybe LabeledProgram
sfiAlgorithm' cur_prog curr =
  if curr >= length cur_prog
    then Just cur_prog
    else
      let (l, curr') = cur_prog !! curr
       in case curr' of
            i@(Store _ dst off _) -> do
              -- In store we want to guard the destination
              (newProg, newCurr) <- handleMemloc l dst off i
              -- After adding the guard, we now replace Reg and off with reg 11
              sfiAlgorithm' newProg (newCurr + 1)
            i@(Load _ _ src off) -> do
              -- In load we want to guard the source
              (newProg, newCurr) <- handleMemloc l src off i
              sfiAlgorithm' newProg (newCurr + 1)
            _ -> sfiAlgorithm' cur_prog (curr + 1)
    where
    ...
    handleMemloc :: Label -> Reg -> Maybe MemoryOffset -> Instruction -> Maybe (LabeledProgram, Int)
    handleMemloc l reg off i =
      -- First create the guard
      let guard = getGuard l reg off i
          -- Remove the original possibly offending statement
          lprog'' = removeLabel l cur_prog
          -- Increment all labels, and make sure our jumps are correct
          fixedProg = newLabels (length guard) l lprog''
          -- Add the guard to the program, and then sort
          newProg = sortOn fst (fixedProg ++ guard)
       in Just (newProg, l + length guard - 1)

    newLabels :: Int -> Label -> LabeledProgram -> LabeledProgram
    newLabels len l = map (updateLabel l len)
    updateLabel :: Label -> Int -> (Int, Instruction) -> (Int, Instruction)
    updateLabel l len (l', instr) =
      -- n is the label the instr wants to jump to (in case of jump)
      let (n, off) = case instr of
            JCond _ _ _ code -> (getLabel code l', code)
            Jmp code -> (getLabel code l', code)
            _ -> (0, 0) -- dummy, won't be used
            -- New target for jump instructions
          newTarget
            -- If we were jumping over the current label, then we now have to
            -- take the guard into account. (Also irrelevant to use >=, as the
            -- current l witll never be a jump anyhow.)
            -- Also Also, in the case we are jumping from somewhere after, we
            -- want to update it to instead jump to the guard. This is why n ==
            -- l is included in the n <=.
            | n <= l && l' >= l = off - (fromIntegral len - 1)
            | n > l && l' <= l = off + (fromIntegral len - 1)
            -- Otherwise, the guard changes nothing.
            | otherwise = off
          newInstr = case instr of
            JCond cmp reg regimm _ ->
              JCond cmp reg regimm (fromIntegral newTarget)
            Jmp _ ->
              Jmp (fromIntegral newTarget)
            i -> i
       in (if l' >= l then l' + len - 1 else l', newInstr)
\end{lstlisting}
The SFI implementation I used is the one mention in Section~\ref{sec:address}.
After we have checked all prerequisites our goal is now to guard all memory
accesses to the specified regions.

The general SFI algorithm can be seen in Listing~\ref{lst:sfi}. We recursively
go through the labeled program and if we ever find a load or store operation
(the only two statements involving memory access), then we handle them via
\texttt{handleMemloc}. This function returns a new program, and the next
instruction we have to check. The function follows the general structure:

\begin{itemize}
  \item First, get the statements that replace the current load or store
    statement. These include the arithmetic in Section~\ref{sec:address}, and
    they also replace the current instruction to instead use our designated
    registers r11 and r9.
  \item Then we remove the current statement from the original program.
  \item We then update all labels in the program. Any label that would have
    come after the original statement has to be incremented by the length of
    the guard we are inserting\footnote{This guard length depends on whether or
    not we have an offset or not. Without an offset, we can save an
  instruction.}. Now all the labels that are in the guard can not be found in
  the original program, and so we are able to insert the guard list of
  statements and sort the labeled program on the Labels to get the new program.
  \item We do not want our sfi algorithm to run on these new inserted
    statements, so we update what label should be checked next to be the
    original label plus the length of the guard minus 1. This gives us the
    label that is right after the original statement in the original program.
\end{itemize}

The \texttt{newLabels} function does the majority of the work here. It both has
to update the current label of all the statement, but it also has to update all
jump instructions to jump to the new locations. Importantly, we want every jump
that jumped to the original load/store instruction to now jump to the start of
our guard, such that we always guarantee that any load adheres to the SFI
policy.
