\section{Conclusion}\label{sec:conclusion}

In this project, I have designed and implemented a Software Fault Isolation
(SFI) policy for eBPF bytecode. The central goal was to enforce memory and
control-flow safety by rewriting eBPF programs to ensure all memory accesses
are confined to designated regions - a stack space and a data region - and that
all jumps are local to code within the program.

My implementation, developed in Haskell, first performs a prerequisite check on
the eBPF program. This static pass verifies that there is no use of reserved
registers are not used, and that all control-flow transfers (jumps) are
directed within the program's boundaries. Following this validation, the core
of my approach involves traversing the program and replacing each memory load
and store instruction with a sequence of guard instructions. I settled on an
address masking technique, which, despite requiring `r2` to be a power of two,
was chosen over a more naive approach to reduce the overhead of conditional
jumps. This technique forces every memory access into a valid region using
bitwise operations, with a single conditional jump to select between the stack
and data regions.

The discussion highlighted several alternative designs and potential
optimizations. While static value analysis proved difficult to apply for
proving safety due to the dynamic nature of the memory region registers, it
could be used to detect guaranteed out-of-bounds accesses. Another significant
optimization would be to eliminate redundant guards by analyzing data flow to
see if a memory location has already been verified.

In conclusion, this work successfully demonstrates a practical method for
enforcing SFI on eBPF programs. The implemented address masking provides a
robust safety guarantee, but with specific constraints on program structure.
