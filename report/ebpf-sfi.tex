\documentclass{article}
\usepackage{amsmath}
\usepackage{amsfonts}
\usepackage{amssymb}
\usepackage{graphicx}
\usepackage{listings}
\usepackage{stmaryrd}

\title{Value Analysis for eBPF}
\author{Simon}
\date{\today}
\lstset{
    basicstyle=\ttfamily\footnotesize,
    breakatwhitespace=false,         
    breaklines=true,                 
    captionpos=b,                    
    keepspaces=true,                 
    numbers=left,                    
    numbersep=5pt,                  
    showspaces=false,                
    showstringspaces=false,
    showtabs=false,                  
    tabsize=2
}

\begin{document}

\maketitle

\begin{abstract}
This report presents a value analysis for a subset of the eBPF language, called
micro-eBPF. The analysis is based on the theory of abstract interpretation,
which uses intervals to approximate the possible values of registers and memory
cells at each program state.
\end{abstract}

\tableofcontents

\section{Introduction}
The SFI policy has two main stipulates. First, any memory access must lie in
either the interval of $[r1, r1 + r2)$ or $[r10, r10+512]$. Second, all
branches must be to the code region where the eBPF code is loaded.

The approach for solving the two varies significantly. The later is something
which can be checked statically before program execution. With a single pass
over the program, we could check every jump instruction and guarantee that it
jumps to somewhere else in the program. For this I decided that the Call
instruction is an instruction which violates the SFI policy, as we technically
would execute code outside the current programs code. 

However, the former is not something which we can conclude at by a simple check
of the program, as we do not know the initial values of the registers. For this
we could use a Inlined Reference Monitor Rewriter as mentioned in \cite{SFI}.
However, as our memory region depends on the registers r1, r2 and r10, then it
seems a logical addition to the SFI policy, that any program which updates the
aforementioned registers violate the SFI policy. Otherwise, any program could
simply overwrite the registers and they would have arbitrary access to memory.
This is another prerequisite (in the same manner as the jump instruction), and
we can check this by a single pass over the program.

For this project I theorized about two primary approaches, which are presented in the following sections.

\subsection{The naive approach}
\begin{lstlisting}[caption={Example program}, label={lst:naive}]
// Example instruction
mem(r1 + 12) := r2

// Rewritten program
r11 := r1 + 12
r9 := r10 + 511
if (r11 <= r10) goto next
if (r11 > r9) goto next
mem(r11) := r2
goto end
next:
  r9 := r1 + r2
  if (r11 <= r1) goto error
  if (r11 > r9) goto error
  mem(r11) := r2
end:
  Exit
\end{lstlisting}
The naive approach to the SFI enforcement of data regions would be to add
conditional jumps whenever we have a memory load/store. In
Listing~\ref{lst:naive}a program following the syntax defined in \cite{SFI} is
presented on how this would work.

This SFI enforcements work on conditional jumps, 4 in total to be exact. First,
we check if we are in the memory region of $[r10, r10 + 512)$, if we are inside
this region, then we fall through and can update the mem(r11) region. However,
if we are not within the region, then we check for $[r1, r1+r2)$ to see if we
are in this memory region. Further, we also impose two new requirements on
registers. As we are using the registers r9 and r11 for computations, then we
would overwrite any data the user has in those registers. As such, when
checking the prerequisites we would also have to check for the use of these two
registers.

However, as this approach uses more conditional jumps than I would like I
decided to move on from it and instead use the next approach.

\subsection{Address masking}
Address masking as presented in \cite{SFI} would require that we impose
restrictions on r1, r2 and r10. Namely we would have to impose that r2 is some
power of two, and we have to impose that r1 fits in the upper bits of r2, i.e
given $r2 - 1 = 0xFFF$, then r1 would have to be $r1=...000$, where ...
indicate that we can have any values come before, and we would need the same
requirement for r10 but in regards to $t11$. If this held, then we would be
able to use $r2-1$ as the address mask, and use a binary OR to then compute the
correct region.

\begin{lstlisting}[caption={Example program}, label={lst:mask}, mathescape=true]
// Example instruction
mem(r1 + 12) := r2

// Rewritten program
r2 := r2 - 1
r9 := r1 + 12
r11 := r9
r9 := r9 - r10
r9 := r9 & 511
r9 := r9 + r10
if (r9 = r11) goto succ
r11 := r11 - r1
r11 := r11 & r2
r11 := r11 + r1
mem(r11) := r2
goto end
succ:
  mem(r9) := r2
end:
  Exit
\end{lstlisting}

Although this might have been a more performant solution than the one I have
applied, I found that this was too imposing on the data region. As such, I have
decided instead only to impose that r2 should be a power of 2. Together with
this I require that r9 and r11 are not used as in the naive approach. This
allows us to rewrite the previous instruction as seen in Listing~\ref{lst:mask}.

\section{Implementation}
\subsection{Prerequisite}
\begin{lstlisting}[language={haskell}, caption={Cheking prerequisites}, label={lst:pre}, escapechar=@]
checkPrerequisite :: ((Int, Instruction) -> Bool) -> LabeledProgram -> Maybe ()
checkPrerequisite p lprog =
  let res = any p lprog
   in (if res then Nothing else Just ())

-- Example use case for checking jump instructions.
checkJumpInstructions :: LabeledProgram -> Maybe ()
checkJumpInstructions lprog = checkPrerequisite checkInst lprog
 where
  checkInst (l, instr) = case instr of
    JCond _ _ _ off -> checkJump off l
    Jmp off -> checkJump off l
    -- Do not allow any calls to extern
    (Call _) -> True
    _ -> False
  checkJump off l =
    let
      res = all (\(i', _) -> i' == getLabel off l) lprog @\label{line:res}@
     in
      not res

checkPrerequisites :: LabeledProgram -> Maybe ()
checkPrerequisites lprog = do
  -- Registers 1,2,10 are restricted by definition of the assignment. register 9, 11 is used for guards
  _ <- checkRegisterUse lprog [1, 2, 9, 10, 11]
  _ <- checkJumpInstructions lprog
  -- Now that we have a program that is nice, we can just return
  return ()
\end{lstlisting}

The first item of the implementation is checking the prerequisites mentioned in
Section~\ref{sec:introduction}. For this I defined the helper function
\texttt{checkPrerequisite}, which can be seen in Listing~\ref{lst:pre}. The
function simply checks if the predicate is true for any of the statements in
the program. If it is, then we return \texttt{Nothing}, otherwise we return \texttt{Just ()}.

An example of the use case is the \texttt{checkJumpInstructions}. Within this
function, we define the predicate \texttt{checkInst}. If we have a jump
condition, then we have to check if there exists some other label in the
labeled program, which starts with the location we are attempting to jump to
(the \texttt{getLabel} takes the offset and the current label, and calculates
the label we would jump to). If the jump is illegal, then res would be False on
line~\ref{line:res}, so we negate res and return.

With this, we can use the functions with the do notation of the Maybe monad as
seen in \texttt{checkPrerequisites} in Listing~\ref{lst:pre}. If any of the
prerequisites return \texttt{Nothing}, then the entire function simply returns
\texttt{Nothing}.

\subsection{SFI Policy Algorithm}
\begin{lstlisting}[language={haskell}, caption={SFI algorithm}, label={lst:sfi}]
sfiAlgorithm' :: LabeledProgram -> Int -> Maybe LabeledProgram
sfiAlgorithm' cur_prog curr =
  if curr >= length cur_prog
    then Just cur_prog
    else
      let (l, curr') = cur_prog !! curr
       in case curr' of
            i@(Store _ dst off _) -> do
              -- In store we want to guard the destination
              (newProg, newCurr) <- handleMemloc l dst off i
              -- After adding the guard, we now replace Reg and off with reg 11
              sfiAlgorithm' newProg (newCurr + 1)
            i@(Load _ _ src off) -> do
              -- In load we want to guard the source
              (newProg, newCurr) <- handleMemloc l src off i
              sfiAlgorithm' newProg (newCurr + 1)
            _ -> sfiAlgorithm' cur_prog (curr + 1)
    where
    ...
    handleMemloc :: Label -> Reg -> Maybe MemoryOffset -> Instruction -> Maybe (LabeledProgram, Int)
    handleMemloc l reg off i =
      -- First create the guard
      let guard = getGuard l reg off i
          -- Remove the original possibly offending statement
          lprog'' = removeLabel l cur_prog
          -- Increment all labels, and make sure our jumps are correct
          fixedProg = newLabels (length guard) l lprog''
          -- Add the guard to the program, and then sort
          newProg = sortOn fst (fixedProg ++ guard)
       in Just (newProg, l + length guard - 1)

    newLabels :: Int -> Label -> LabeledProgram -> LabeledProgram
    newLabels len l = map (updateLabel l len)
    updateLabel :: Label -> Int -> (Int, Instruction) -> (Int, Instruction)
    updateLabel l len (l', instr) =
      -- n is the label the instr wants to jump to (in case of jump)
      let (n, off) = case instr of
            JCond _ _ _ code -> (getLabel code l', code)
            Jmp code -> (getLabel code l', code)
            _ -> (0, 0) -- dummy, won't be used
            -- New target for jump instructions
          newTarget
            -- If we were jumping over the current label, then we now have to
            -- take the guard into account. (Also irrelevant to use >=, as the
            -- current l will never be a jump anyhow.)
            -- Also Also, in the case we are jumping from somewhere after, we
            -- want to update it to instead jump to the guard. This is why n ==
            -- l is included in the n <=.
            | n <= l && l' >= l = off - (fromIntegral len - 1)
            | n > l && l' <= l = off + (fromIntegral len - 1)
            -- Otherwise, the guard changes nothing.
            | otherwise = off
          newInstr = case instr of
            JCond cmp reg regimm _ ->
              JCond cmp reg regimm (fromIntegral newTarget)
            Jmp _ ->
              Jmp (fromIntegral newTarget)
            i -> i
       in (if l' >= l then l' + len - 1 else l', newInstr)
\end{lstlisting}
The SFI implementation I used is the one mention in Section~\ref{sec:address}.
After we have checked all prerequisites our goal is now to guard all memory
accesses to the specified regions.

The general SFI algorithm can be seen in Listing~\ref{lst:sfi}. We recursively
go through the labeled program and if we ever find a load or store operation
(the only two statements involving memory access), then we handle them via
\texttt{handleMemloc}. This function returns a new program, and the next
instruction we have to check. The function follows the general structure:

\begin{itemize}
  \item First, get the statements that replace the current load or store
    statement. These include the arithmetic in Section~\ref{sec:address}, and
    they also replace the current instruction to instead use our designated
    registers r11 and r9.
  \item Then we remove the current statement from the original program.
  \item We then update all labels in the program. Any label that would have
    come after the original statement has to be incremented by the length of
    the guard we are inserting\footnote{This guard length depends on whether or
    not we have an offset or not. Without an offset, we can save an
  instruction.}. Now all the labels that are in the guard can not be found in
  the original program, and so we are able to insert the guard list of
  statements and sort the labeled program by the labels to get the new program.
  \item We do not want our sfi algorithm to run on these new inserted
    statements, so we update what label should be checked next to be the
    original label plus the length of the guard minus 1. This gives us the
    label that is right after the original statement in the original program.
\end{itemize}

The \texttt{newLabels} function does the majority of the work here. It has to
update both the current label of all the statements and all jump instructions
to jump given these new labels. Importantly, we want every jump that jumped to
the original load/store instruction to now jump to the start of our guard, such
that we always guarantee that any load adheres to the SFI policy.

\subsection{Validation}
The implementation was mainly validated with manual programs and then checked
that they produced the expected result. These manual tests included a lot of
jump instructions that jump differently over the different load and store
instructions to make sure that we specifically update the jumps correctly.

Further, some simple tests were added for the prerequisites.

\input{sections/validation}
% Discuss why not use naive approach first, and then force address masking. Want to reduce the conditional jumps.

\section{Conclusion}\label{sec:conclusion}

In this project, I have designed and implemented a Software Fault Isolation
(SFI) policy for eBPF bytecode. The central goal was to enforce memory and
control-flow safety by rewriting eBPF programs to ensure all memory accesses
are confined to designated regions - a stack space and a data region - and that
all jumps are local to code within the program.

My implementation, developed in Haskell, first performs a prerequisite check on
the eBPF program. This static pass verifies that reserved registers are not
used, and that all control-flow transfers (jumps) are directed within the
program's boundaries. Following this validation, the core
of my approach involves traversing the program and replacing each memory load
and store instruction with a sequence of guard instructions. I settled on an
address masking technique, which, despite requiring r2 to be a power of two,
was chosen over a more naive approach to reduce the overhead of conditional
jumps. This technique forces every memory access into a valid region using
bitwise operations, with a single conditional jump to select between the stack
and data regions.

The discussion highlighted several alternative designs and potential
optimizations. While static value analysis proved difficult to apply for
proving safety due to the dynamic nature of the memory region registers, it
could be used to detect guaranteed out-of-bounds accesses. Another significant
optimization would be to eliminate redundant guards by analyzing data flow to
see if a memory location has already been verified.

In conclusion, this work successfully demonstrates a practical method for
enforcing SFI on eBPF programs. The implemented address masking provides a
robust safety guarantee, but with specific constraints on program structure.



\bibliographystyle{plain}   % or try alpha, unsrt, abbrv
\bibliography{refs}         % refs.bib file

\end{document}
